\def\XCfileversion{v0.1}%
\def\XCfiledate{2014-09-01}%
\documentclass[letter]{article}
\usepackage{array,multicol,multido,textcomp,amsmath,graphicx,hyperref,subfigure,underscore,parskip}
\usepackage{xcolor}[2005/03/24]
\usepackage[normalem]{ulem}
\usepackage[hmargin={1.25cm,.75cm},vmargin=1.25cm,footskip=.5cm,nohead]{geometry}
\begin{document}
\title{White Oaks Institute - Mathematics Worksheet}
\author{Arithmetic\thanks{This file was generated by an \textsf{Open Source} mathematics worksheet program which can be downloaded at \texttt{www.metaciv.org}. Please send error reports and suggestions for improvements to \texttt{wneal@lia-edu.ca}.}}
\date{\XCfileversion{} (\XCfiledate)}
\maketitle
\setlength{\parskip}{12mm plus 4mm minus 4mm}\setlength{\parindent}{0cm}\begin{multicols}{2}(1) $ 9\times2= $\hspace{3 mm}\line(1,0){125}\\\\(2) $ 7\times10= $\hspace{3 mm}\line(1,0){125}\\\\(3) $ 4\times6= $\hspace{3 mm}\line(1,0){125}\\\\(4) $ 8\times10= $\hspace{3 mm}\line(1,0){125}\\\\(5) $ 3\times5= $\hspace{3 mm}\line(1,0){125}\\\\(6) $ 2\times9= $\hspace{3 mm}\line(1,0){125}\\\\(7) $ 3\times3= $\hspace{3 mm}\line(1,0){125}\\\\(8) $ 6\times6= $\hspace{3 mm}\line(1,0){125}\\\\(9) $ 4\times6= $\hspace{3 mm}\line(1,0){125}\\\\(10) $ 6\times10= $\hspace{3 mm}\line(1,0){125}\\\\(11) $ 3\times4= $\hspace{3 mm}\line(1,0){125}\\\\(12) $ 10\times7= $\hspace{3 mm}\line(1,0){125}\\\\(13) $ 4\times2= $\hspace{3 mm}\line(1,0){125}\\\\(14) $ 5\times10= $\hspace{3 mm}\line(1,0){125}\\\\(15) $ 3\times8= $\hspace{3 mm}\line(1,0){125}\\\\(16) $ 5\times7= $\hspace{3 mm}\line(1,0){125}\\\\(17) $ 10\times10= $\hspace{3 mm}\line(1,0){125}\\\\(18) $ 3\times2= $\hspace{3 mm}\line(1,0){125}\\\\(19) $ 2\times2= $\hspace{3 mm}\line(1,0){125}\\\\(20) $3\times2= $\line(1,0){125}\end{multicols}\end{document}


